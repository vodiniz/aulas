\documentclass{article}

\usepackage[margin=1in]{geometry}

\usepackage{amsmath}
\usepackage{amssymb}
\usepackage{calc}
\usepackage{enumitem}
\usepackage{lmodern}
\usepackage{pgfplots}
\usepackage{qtree}
\usepackage{slashed}
\usepackage{upgreek}
\usepackage{xfrac}
\usepackage[normalem]{ulem}
\usepackage[utf8]{inputenc}
\usepackage[T1]{fontenc}
\usepackage[hidelinks]{hyperref}

\begin{document}

\section{Matrizes}

\subsection{Matriz}

Uma matriz $A, m \times n( m$ por $n$) é uma tabela de $mn$ números dispostos em $m$ linhas e $n$ colunas

\begin{equation*}
    A_{m,n} = 
    \begin{bmatrix}
    a_{11} & a_{12} & \cdots & a_{1n} \\
    a_{21} & a_{22} & \cdots & a_{2n} \\
    \vdots  & \vdots  & \ddots & \vdots  \\
    a_{m1} & a_{m2} & \cdots & a_{mn} 
    \end{bmatrix}
\end{equation*}

A $i-$\textbf{ésima  linha }de A é 

\begin{equation*}
    \begin{bmatrix}
        a_{i1} & a_{i2} & \cdots & a_{in} \\
    \end{bmatrix}
\end{equation*}

A $j-$\textbf{ésima coluna} de A é 

\begin{equation*}
    \begin{bmatrix}
        a_{1j} \\
        a_{2j} \\
        \vdots \\
        a_{mj} 
    \end{bmatrix}
\end{equation*}

Se $m = n$ \bigbreak
dizemos que A é uma \textbf{matriz quadrada de ordem} $n$ e os elementos \newline
\begin{center}
    $ a_{11}, a_{22}, \hdots a_{nn}$
\end{center}
formam a \textbf{diagonal (princinpal)} de $A$.

\subsection{Exemplos}

\begin{equation*}
    A_{2 \times 2} = 
    \begin{bmatrix}
    1 & 2 \\
    3 & 4 \\
    \end{bmatrix}
\end{equation*}
$$ a_{12} = 2$$ $$ [A]_{22} = 4$$
\newline
\begin{equation*}
    B_{2 \times 2} = 
    \begin{bmatrix}
    2- & 1 \\
    0 & 3 \\
    \end{bmatrix}
\end{equation*}
\newline
\begin{equation*}
    C_{2\times3} = 
    \begin{bmatrix}
    1 & 3 & 0 \\
    2 & 4 & -2 \\
    \end{bmatrix}
\end{equation*}
$$c_{23} = -2 $$
\newline
\begin{equation*}
D_{1 \times 3} = 
    \begin{bmatrix}
    1 & 3 & -2 \\
    \end{bmatrix}
\end{equation*}
$$[D]_{12} = -3 $$
\begin{center}
    \textbf{matriz linha} 
\end{center}


\begin{equation*}
E_{1 \times 3} = 
    \begin{bmatrix}
    1 \\
    4 \\
    -3 \\
    \end{bmatrix}
\end{equation*}
$$ e_{21} = 4$$
\begin{center}
    \textbf{matriz coluna} 
\end{center}

\begin{equation*}
    F_{1 \times 1} = 
        \begin{bmatrix}
        3 \\
        \end{bmatrix}
\end{equation*}
\newline

$ A = (a_{ij})_{m \times n} $\ e \ $ B = (b_{ij})_{p \times q} $ são \textbf{iguais} se: \newline
\begin{center}
    $$ m = p $$
    $$ n = q $$
    $$ a_{ij} = b_{ij} $$
\end{center}

\subsection{Soma de Matrizes}
\subsubsection{Definição}
 A \textbf{soma} de dias matrizes de \textbf{mesmo tamanho} $A = (a_{ij})_{m \times n}$
 é definida como sendo a matriz $m \ x \ n$.
 $$C = A + B$$
 obtida somando-se os elementos correspondes de $A$ e $B$, ou seja \bigbreak
 $ c_{ij} = a_{ij} + b_{ij}$, \bigbreak
 para $ i = 1, \hdots, m $ e $j = 1, \hdots, n $. Escrevemos também $[A + B]_{ij} = a_{ij} + b_{ij}$.

\subsubsection{Exemplo}
Considere as matrizes:

\begin{equation}
    A = 
    \begin{bmatrix}
    1 & 2 & -3 \\
    3 & 4 & 0 \\
    \end{bmatrix}, \
    B = 
    \begin{bmatrix}
    -2 & 1 & 5 \\
    0 & 3 & -4 \\
    \end{bmatrix}
\end{equation}
 Se chamamos de $C$ a soma das duas matrizes $A$ e $B$, então :
 \begin{equation}
    C = A + B = 
    \begin{bmatrix}
        1 + (-2) & 2+1 & -3+5 \\
        3+0 & 4+3 & 0+(-4)
    \end{bmatrix}
    = 
    \begin{bmatrix}
        -1 & 3 & 2 \\
        3 & 7 & -4 \\
    \end{bmatrix}
 \end{equation}

 \section{Multiplicação de Matrizes}
\subsubsection{Definição}
 A \textbf{multiplicação de uma matriz} $A = (a_{ij})_{m \times n}$ \textbf{por um escalar} (número) $\alpha$ é definida pela matriz $m \times n$
 $$B = \alpha A$$
 obtida multiplicando-se cada elemento da matriz $A$ pelo escalar $\alpha $, ou seja,
 $$b_{ij} = \alpha a_{ij}$$

 \subsubsection{Exemplo}

 O produto da matriz $ A = 
 \begin{bmatrix}
    -2 & 1 \\
    0 & 3 \\
    5 & 4 \\
 \end{bmatrix}$
 pelo escalar $-3$ é dado por :
 $$-3\ A =  \begin{bmatrix}
    (-3)(-2) & (-3)(1) \\
    (-3)0 & (-3)3 \\
    (-3)5 & (-3)(-4) \\
 \end{bmatrix}
 =
 \begin{bmatrix}
    6 & -3 \\
    0 & 9 \\
    -15 & 12
\end{bmatrix}
.
 $$

 \subsubsection{Definição}
 O produto de duas matrizes, tais que o \textbf{o número de colunas da primeira matriz é igual ao número de linhas da segunda},
 $ A = (a_{ij})_{m \times p}$ e $B = (b_{ij})_{p \times n}$ é definido pela matriz $m \times n$

$$C = AB$$
obtida da seguinte forma:

$$c_{ij} = a_{i1}b_{1j} + a_{i2}b_{2j} + \hdots + a_{ip}b_{pj}$$

para $i = 1, \hdots, m$ e $j = 1, \hdots, n$. Escrevemos tambem $[AB]_{ij} = a_{i1}b_{1j} + a_{i2}b_{2j} + \hdots + a_{ip}b_{pj}$
\begin{equation}
    \begin{bmatrix}
        c_{11} & \hdots & c_{1n} \\
        \vdots & c_{ij} & \vdots \\
        c_{m1} & \hdots & c_{mn} \\
    \end{bmatrix}
    =
    \begin{bmatrix}
        a_{11} & a_{12} & \hdots & a_{1p} \\
        \vdots & & \hdots & \vdots \\
        a_{i1} & a_{i2} & \hdots & a_{ip} \\
        a_{m1} & a_{m2} & \hdots & a_{mp} \\
    \end{bmatrix}
    \begin{bmatrix}
        b_{11} & \hdots & b_{1j} & \hdots & b_{1n} \\
        b_{21} & \hdots & b_{2j} & \hdots & b_{2n} \\
        \vdots & & \hdots & \vdots & \\
        b_{p1} & \hdots & b_{p2} & \hdots & b_{pn} & \\
    \end{bmatrix}
\end{equation}

\subsubsection{Exemplo}
Considere as matrizes
\begin{equation*}
    A = 
    \begin{bmatrix}
        1 & 2 & -3 \\
        3 & 4 & 0 \\
    \end{bmatrix},
    B = 
    \begin{bmatrix}
        -2 & 1 & 0 \\
        0 & 3 & 0 \\
        5 & -4 & 0 \\
    \end{bmatrix},
\end{equation*}
Se chamamos de C o produto das duas matrizes $A$ e $B$, então :
\begin{equation*}
    C = AB =
    \begin{bmatrix}
        1(-2) +2 \cdot 0 +(-3)5 & 1 \cdot 1 +2 \cdot 3 +(-3)(-4) & 0 \\
        3(-2) +4 \cdot 0 + 0 \cdot 5 & 3 \cdot 1 + 4 \cdot 3 + 0(-4) & 0 \\
    \end{bmatrix}
    =
    \begin{pmatrix}
        -17 & 19  & 0 \\
        -6 & 15 & 0 \\
    \end{pmatrix}
\end{equation*}

\noindent\makebox[\linewidth]{\rule{\paperwidth}{0.8pt}}
\textbf{Observação.} No exemplo anterior o produto $BA$ não está definido (porque ?). Entretanto, mesmo quando ele está definido,
    pode não ser igual a AB, ou seja, o produto de matrizes \textbf{não é comutativo}, como mostra o exemplo seguinte
    \newline
    \noindent\makebox[\linewidth]{\rule{\paperwidth}{0.8pt}}


\subsubsection{Exemplo}
Sejam $
    A = 
    \begin{bmatrix}
        1 & 2 \\
        3 & 4 \\
    \end{bmatrix},$ e $
    B = 
    \begin{bmatrix}
        -2 & 1 \\
        0 & 3 \\
    \end{bmatrix}.
    $ Então, $
    AB =
    \begin{bmatrix}
        -2 & 7 \\
        -6 & 15 \\
    \end{bmatrix} 
    $, e $
    BA = 
    \begin{bmatrix}
        1 & 0 \\
        9 & 12 \\
    \end{bmatrix} 
    $

\section{Transposta}
    A \textbf{transposta} de uma matriz $A = (A_{ij})_{m \times n}$ é definida pela matriz $m \times n$.
    $$B = A^t$$ 
    $$[A^t]_{ij} = a_{ji}$$

    obtida trocando-se as linhas com as colunas, ou seja:
    $$b_{ij} = a_{ji}$$


    \begin{equation*}
        A = 
        \begin{bmatrix}
            1 & 2 \\
            3 & 4 \\
        \end{bmatrix} \ \ \ \
        A^t = 
        \begin{bmatrix}
            1 & 3 \\
            2 & 4 \\
        \end{bmatrix}
    \end{equation*}
    \newline
    \begin{equation*}
        C = 
        \begin{bmatrix}
            1 & 2 & 0\\
            3 & 4 & -2\\
        \end{bmatrix} \ \ \ \
        C^t = 
        \begin{bmatrix}
            1 & 2 \\
            3 & 4 \\
            0 & -2 \\
        \end{bmatrix}
    \end{equation*}


\section{Teoremas}
Sejam $A$, $B$ e $C$ matrizes com tamanhos apropriados, $\alpha$ e $\beta$ escalares. São 
válidas as seuintes propriedades para as operações matriciais
\begin{enumerate}[label=(\alph*)]
    \item (Comutatividade) $A + B = B + A$;
    \item (Associatividade) $A + (B + C) = (A + B) + C$;
    \item (Elemento Neutro) A matriz $\overline{0}$ definida por $[\overline{0}]_{ij} = 0$, para $ i = 1,\hdots ,m \ ,j = 1,\hdots ,n$ é tal que
    $$ A + \overline{0} = A, $$
    para toda matriz $A, m \times n$. A matriz $\overline{0}$ é chama de \textbf{matriz nula} $m \times n$
    \item (Elemento Simétrico) Para cada matriz $A$, existe uma única matriz $-A$, definida por $[-A]_{ij} = -a_{ij}$, tal que 
    $$ A + (-A) = \overline{0}. \ \ \ \ \ \ \ A -B = A + (-B)$$ ;
    \item (Associatividade) $\alpha(\beta A) = (\alpha \beta )A$;
    \item (Distributividade) $(\alpha + \beta)A = \alpha A + \beta B $;
    \item (Distributividade) $\alpha(A + B) = \alpha A + \beta B $;
    \item (Associatividade) $A(BC) = (AB)C $; \ \ \ \ \ \ \ \ 
    $\underbrace{A^P = A \hdots}_{p vezes}$
    \item (Elemento Neutro) Para cada inteiro positivo $p$ a matriz $p \times p$,
    $$I_p = 
    \begin{bmatrix}
        1 & 0 & \hdots & 0 \\
        0 & 1 & \hdots & 0 \\
        \vdots & & \ddots & \vdots \\
        0 & 0 & \hdots & 1 \\
    \end{bmatrix}
    , \ \ \ \ \ \ \ \ \ A^0 = I^n
    $$
    chamada \textbf{matriz identidade} é tal que \newline
    $A i_n = I_mA = A$, para toda matriz $A = (a_{ij})_{m \times n}$.
    \item (Distributividade) $A(B+C) = AB + AC $ e $(B + C)A = BA + CA$;
    \item $\alpha(AB) = (\alpha A)B = A(\alpha B)$;
    \item $(A^t)^t = A$;
    \item $(A + B)^t = A^t + B^t$;
    \item $(\alpha A)^t = \alpha A^T$
    \item $(AB)^t = B^t A^t$;

\end{enumerate}

\section{Matriz Diagonal}

 \textbf{Matriz diagonal} $n \times n$
 $ D = 
 \begin{bmatrix}
    \lambda_1 & 0 & \hdots & 0 \\
    0 &  \lambda_2 & \hdots & 0 \\
    \vdots & & \ddots & \vdots \\
    0 & \hdots & 0 & \lambda_n \\
\end{bmatrix}
 $ A é chamada de \textbf{nilpotente} se $A^k = \overline{0}$, para algum inteiro positivo $k$
$$
A = \begin{bmatrix}
    0 & 1 & 0 & \hdots & 0 \\
    0 & 0 & 1 & \hdots & 0 \\
    \vdots & \vdots & \ddots & \ddots & \vdots \\
    0 & 0 & 0 & \hdots & 1 \\
    0 & 0 & 0 & \hdots & 0 \\
\end{bmatrix}_{n \times n}
$$

\section{Matriz Simétrica}

$A, n \times n$, é \textbf{simétrica} se $A^t = A$ e \textbf{anti-simétrica} se $A^t = -A$.
Se A é simétrica, então $a_{ij} = a_{ji}$ \newline
Se $A$ e $B$ são simétricas, então $A + B$ e $\alpha A$ são simétrica \newline
Se A e B são simétricas, então AB é simétrica se, e somente se, $AB = BA$


 \end{document}



