\documentclass{article}

\usepackage[margin=1in]{geometry}

\usepackage{amsmath}
\usepackage{amssymb}
\usepackage{arydshln}
\usepackage{calc}
\usepackage{enumitem}
\usepackage{lmodern}
\usepackage{mathtools}
\usepackage{pgfplots}
\usepackage{qtree}
\usepackage{slashed}
\usepackage{upgreek}
\usepackage{xfrac}
\usepackage[normalem]{ulem}
\usepackage[utf8]{inputenc}
\usepackage[T1]{fontenc}
\usepackage[hidelinks]{hyperref}

\begin{document}

\section{Sobre as Matrizes}

Todo numero real $a$, não nulo possui um inverso ( multiplicativo ), ou seja existe um número $b$,
 tal que $a\ b = b\ a = 1$. Este número é único e o denotamos por $a^{-1}$. 
 Apesar da álgebra matricial ser semalhante à álgebra dos números reais, nem todas as matrizes
  $A$ \textit{não nulas} possuem inversa, ou seja, nem sempre existe uma matriz $B$, tal que $A\ B = B\ A = I_n$.
  De início, para que os produtos $AB$ e $BA$, estejam definidos e sejam iguais é preciso que 
  as matrizes $A$ e $B$ sejam quadradas. Portanto, somente as matrizes quadradas podem ter inversa, o que já diferencia
  do caso dos números reais, pois todo núymero não nulo tem inverso. Mesmo que entre as matrizes quadradas, muitas não possuem inversa, 
  apesar do conjunto das que não tem inversa ser bem menor do que o conjunto das que tem.
   

\section{Matriz Inversa}
\subsection{Definição}
Uma matriz quadrada $A = (A_{ij})_{n \times m}$ é \textbf{invertivel}, ou \textbf{não singular}, se existe uma matriz
$B = (b_{ij})_{n \times m}$ tal que
$$A \ B = B \ A = I_n,$$
em que $I_n$ é a matriz identidade. A matriz $B$ é chamada de \textbf{inversa} de $A$. Se $A$ não tem inversa,
 dizemos que $A$ é \textbf{não invertível} ou \textbf{singular}
\subsection{Exemplo}
Considere as matrizes \bigbreak
$
  A = 
  \begin{bmatrix}
    -2 & 1 \\
    0 & 3 \\ 
  \end{bmatrix}
$ e 
$
B = 
\begin{bmatrix}
\frac{-1}{2} & \frac{1}{6} \\
0 & \frac{1}{3}
\end{bmatrix}
$
\bigbreak
A matriz $B$ é a inversa da matriz A, pois $A \ B = B \ A = I_2$

\bigbreak

$A \dot B = 
\begin{bmatrix}
  1 & 0 \\
  0 & 1 \\
\end{bmatrix}$

\subsubsection{Teorema}
Sejam $A$ e $B$ matrizes $n \times n$.
\begin{enumerate}[label=(\alph*)]
  \item Se $BA = I_n$, então $AB = I_n$
  \item Se $AB = I_n$, então $BA = I_n$ 
\end{enumerate}

\section{Teoremas}
\subsection{}
Se a matriz $A = (a_{ij})_{n \times n}$ possui inversa, então a inversa é única
\subsection{}
\begin{enumerate}[label=(\alph*)]
  \item Se A é invertivel, então $A^{-1}$ tambem o é e 
  $$(A^{-1})^{-1} = A;$$
  \item Se $A = (a_{ij})_{n \times n}$ e $B = (b_{ij})_{n \times n}$ são matizes invertíveis, então $AB$ é invertivel e 
  $$(AB)^{-1} = B^{-1}A^{-1};$$
  \item Se $A = (a_{ij})_{n \times n}$ é invertivel, então $A^t$ tambem é invertível e 
  $$(A^t)^{-1} = (A^-1)^t$$ 
\end{enumerate}

\section{Operação Elementar sobre as linhas}
\subsection{Definição}
Uma \textbf{operação elementar sobre as linhas} de uma matriz é uma das seguintes operações;
\begin{enumerate}[label=(\alph*)]
  \item Trocar a posição de duas linhas da matriz;
  \item Multiplicar uma linha da matriz por um escalar diferente de zero;
  \item Somar a uma linha da matriz um múltiplo escalar de outra linha.
\end{enumerate} 
\subsection{Definição}
Uma matriz $A = (a_{ij})_{m \times n}$ \textbf{é equivalente por linhas} a uma matriz $B = (b_{ij})_{m \times n}$, se $B$ pode ser 
obtida de $A$ aplicando-se uma sequencia de operações elementares sobre as suas linhas.

\subsection{Teorema}
Seja $A$ uma matriz $n \times n$. As seguintes afirmações são equivalentes:
\begin{enumerate}[label=(\alph*)]
  \item Existe uma matriz $B, n \times n$ tal que $BA = I_n$
  \item A matriz $A$ é equivalente por linhas à matriz identidade $I_n$
  \item A matriz $A$ é invertível.
\end{enumerate}

\subsection{Exemplo}
Vamos encontrar, se existir a inversa de 
$$A = 
\begin{bmatrix}
  1 & 1 & 1 \\
  2 & 1 & 4 \\
  2 & 3 & 5 \\
\end{bmatrix}$$
\subsubsection*{1ª Eliminação}
-2x1ª linha + 2ª linha $\longrightarrow $ 2ª linha \newline
-2x1ª linha + 3ª linha $\longrightarrow $ 3ª linha \newline

$$
  \setlength{\dashlinegap}{2pt}
\left[\begin{array}{ccc:ccc}
  1 & 1 & 1 & 1 & 0 & 0 \\
  0 & -1 & 2 & -2 & 0 & 0 \\
  0 & 1 & 3 & -2 & 0 & 1 \\
  \end{array}
\right]
$$

\subsubsection*{2ª Eliminação}
-1x2ª linha  $\longrightarrow $ 2ª linha \newline

$$
  \setlength{\dashlinegap}{2pt}
\left[\begin{array}{ccc:ccc}
  1 & 1 & 1 & 1 & 0 & 0 \\
  0 & 1 & -2 & 2 & -1 & 0 \\
  0 & 1 & 3 & -2 & 0 & 1 \\
  \end{array}
\right]
$$

-1x2ª linha + 1ª linha $\longrightarrow $ 1ª linha \newline
-1x2ª linha + 3ª linha $\longrightarrow $ 3ª linha \newline

$$
\setlength{\dashlinegap}{2pt}
\left[\begin{array}{ccc:ccc}
  1 & 0 & 3 & -1 & 1 & 0 \\
  0 & 1 & -2 & 2 & -1 & 0 \\
  0 & 0 & 5 & -4 & 1 & 1 \\
  \end{array}
\right]
$$

\subsubsection*{3ª Eliminação}

$\frac{1}{5} \times 3ª$ª linha

$$
  \setlength{\dashlinegap}{2pt}
\left[\begin{array}{ccc:ccc}
  1 & 0 & 3 & -1 & 1 & 0 \\
  0 & 1 & -2 & 2 & -1 & 0 \\
  0 & 0 & 1 & -\frac{4}{5} & \frac{1}{5} & \frac{1}{5} \\
  \end{array}
\right]
$$

-3x3ª linha + 1ª linha $\longrightarrow $ 1ª linha \newline
2x3ª linha + 2ª linha $\longrightarrow $ 2ª linha \newline

$$
  \setlength{\dashlinegap}{2pt}
\left[\begin{array}{ccc:ccc}
  1 & 0 & 0 & \frac{7}{5} & \frac{2}{5}  & -\frac{3}{5}  \\
  0 & 1 & 0 & \frac{2}{5}  & -\frac{3}{5}  & \frac{2}{5}  \\
  0 & 0 & 1 & -\frac{4}{5} & \frac{1}{5} & \frac{1}{5} \\
  \end{array}
\right]
$$

\subsection{Exemplo}
Vamos determinar, se existir a inversa da matriz
$$A = 
\begin{bmatrix}
  1 & 2 & 3 \\
  1 & 1 & 2 \\
  0 & 1 & 1 \\
\end{bmatrix}$$

Para isso devemos escalonar a matriz aumentada

$$
[A | I_3] = 
\setlength{\dashlinegap}{2pt}
\left[\begin{array}{ccc:ccc}
  1 & 2 & 3 & 1 & 0 & 0 \\
  1 & 1 & 2 & 0 & 1 & 0 \\
  0 & 1 & 1 & 0 & 0 & 1 \\
  \end{array}
\right]
$$

\subsubsection*{1ª eliminação:}
-1x1ª linha + 2ª linha $\longrightarrow $ 2ª linha \newline

$$
[A | I_3] = 
\setlength{\dashlinegap}{2pt}
\left[\begin{array}{ccc:ccc}
  1 & 2 & 3 & 1 & 0 & 0 \\
  0 & 1 & 1 & 1 & -1 & 0 \\
  0 & 1 & 1 & 0 & 0 & 1 \\
  \end{array}
\right] 
$$

\subsubsection*{2ª Eliminação}
-1x2ª linha $\longrightarrow $ 2ª linha \newline

$$
[A | I_3] = 
\setlength{\dashlinegap}{2pt}
\left[\begin{array}{ccc:ccc}
  1 & 2 & 3 & 1 & 0 & 0 \\
  0 & 1 & 1 & 1 & -1 & 0 \\
  0 & 1 & 1 & 0 & 0 & 1 \\
  \end{array}
\right] 
$$

-2x2ª linha + 1ª linha $\longrightarrow $ 1ª linha \newline
-1x2ª linha + 3ª linha $\longrightarrow $ 3ª linha \newline

$$
[A | I_3] = 
\setlength{\dashlinegap}{2pt}
\left[\begin{array}{ccc:ccc}
  1 & 0 & 1 & -1 & 2 & 0 \\
  0 & 1 & 1 & 1 & -1 & 0 \\
  0 & 0 & 0 & -1 & 1 & 1 \\
  \end{array}
\right] 
$$

\end{document}